\documentclass[11pt,a4paper]{article}

% ── Paquetes ──
\usepackage[utf8]{inputenc}
\usepackage[T1]{fontenc}
\usepackage[spanish]{babel}
\usepackage{geometry}
\usepackage{graphicx}
\usepackage{xcolor}
\usepackage{listings}
\usepackage{hyperref}
\usepackage{enumitem}
\usepackage{booktabs}
\usepackage{longtable}
\usepackage{fancyhdr}
\usepackage{titlesec}

\geometry{margin=2.5cm}

% ── Colores personalizados ──
\definecolor{hubblue}{HTML}{1A237E}
\definecolor{codebg}{HTML}{F5F5F5}
\definecolor{codeframe}{HTML}{CCCCCC}
\definecolor{sectioncolor}{HTML}{283593}
\definecolor{linkcolor}{HTML}{1565C0}

% ── Hyperlinks ──
\hypersetup{
    colorlinks=true,
    linkcolor=hubblue,
    urlcolor=linkcolor,
    citecolor=hubblue
}

% ── Estilos de sección ──
\titleformat{\section}{\Large\bfseries\color{sectioncolor}}{}{0pt}{}[\titlerule]
\titleformat{\subsection}{\large\bfseries\color{hubblue}}{}{0pt}{}

% ── Estilo de código ──
\lstset{
    backgroundcolor=\color{codebg},
    frame=single,
    rulecolor=\color{codeframe},
    basicstyle=\ttfamily\small,
    breaklines=true,
    breakatwhitespace=true,
    tabsize=4,
    showstringspaces=false,
    keywordstyle=\color{blue}\bfseries,
    commentstyle=\color{gray}\itshape,
    stringstyle=\color{red!70!black},
    language=Java
}

% ── Header/Footer ──
\pagestyle{fancy}
\fancyhf{}
\fancyhead[L]{\small\textcolor{hubblue}{GameHub --- Documentación Técnica Breve}}
\fancyhead[R]{\small\thepage}
\fancyfoot[C]{\small\textcolor{gray}{Proyecto Android Multi-Módulo}}
\renewcommand{\headrulewidth}{0.4pt}

% ══════════════════════════════════════════════════════════════
\begin{document}

\begin{titlepage}
    \centering
    \vspace*{3cm}
    {\Huge\bfseries\textcolor{hubblue}{GameHub}\par}
    \vspace{0.5cm}
    {\Large Documentación Técnica Breve\par}
    \vspace{1.5cm}
    {\large Proyecto Android Multi-Módulo\par}
    \vspace{0.3cm}
    {\normalsize Lenguaje: Java \quad|\quad Build System: Gradle (Kotlin DSL)\par}
    \vspace{0.3cm}
    {\normalsize Min SDK: 26 \quad|\quad Target SDK: 36\par}
    \vspace{2cm}
    {\large\textbf{Módulos:}\par}
    \vspace{0.3cm}
    \begin{tabular}{ll}
        \textbullet\ \texttt{app}                     & Hub principal (login, perfil, menús) \\
        \textbullet\ \texttt{kaisenclicker\_module}    & Juego: Kaisen Clicker \\
        \textbullet\ \texttt{2048\_module}             & Juego: 2048 \\
    \end{tabular}
    \vfill
    {\small Febrero 2026\par}
\end{titlepage}

\tableofcontents
\newpage

% ══════════════════════════════════════════════════════════════
\section{Visión General del Proyecto}

GameHub es una aplicación Android multi-módulo que funciona como un \textbf{centro de juegos}. El módulo principal (\texttt{app}) contiene el sistema de autenticación, gestión de perfil, puntuaciones y desafíos. Los juegos se implementan como módulos independientes que se integran en el hub.

\begin{itemize}[leftmargin=1.5cm]
    \item \textbf{Hub (\texttt{app})}: Login/Registro, pantalla principal con grid de juegos, perfil de usuario, leaderboard, sistema de desafíos.
    \item \textbf{Kaisen Clicker (\texttt{kaisenclicker\_module})}: Juego clicker con temática de Jujutsu Kaisen. Incluye campaña, tienda, cofres, inventario de personajes y estadísticas.
    \item \textbf{2048 (\texttt{2048\_module})}: Implementación del juego clásico 2048 con modos de juego (normal, blitz) y temas visuales.
\end{itemize}

% ══════════════════════════════════════════════════════════════
\section{Base de Datos --- Arquitectura de Persistencia}

La persistencia del proyecto se basa en \textbf{SQLite nativo} (sin Room) gestionado a través de \texttt{SQLiteOpenHelper}. Existen dos bases de datos principales y un sistema auxiliar de \texttt{SharedPreferences}.

\subsection{Base de Datos Principal: \texttt{kaisen\_clicker.db}}

Gestionada por \texttt{AppDatabaseHelper} (versión 4). Es la BD \textbf{global} de toda la aplicación. Almacena datos tanto de Kaisen Clicker como de 2048.

\vspace{0.3cm}
\noindent\textbf{Ruta:} \texttt{kaisenclicker\_module/src/main/java/com/example/kaisenclicker/persistence/save/AppDatabaseHelper.java}

\subsubsection*{Tablas de la BD principal}

\begin{longtable}{p{3cm}p{5cm}p{6cm}}
    \toprule
    \textbf{Tabla} & \textbf{Columnas principales} & \textbf{Descripción} \\
    \midrule
    \endhead
    \texttt{users} & \texttt{id}, \texttt{username} (UNIQUE), \texttt{password\_hash}, \texttt{created\_at} & Usuarios registrados. Contraseñas almacenadas con hash SHA-256. \\
    \midrule
    \texttt{kv\_store} & \texttt{k} (PK), \texttt{value\_text}, \texttt{value\_int}, \texttt{value\_long}, \texttt{value\_real} & Almacén clave-valor genérico. Guarda energía maldita, niveles de mejora, progreso del 2048, etc. \\
    \midrule
    \texttt{characters} & \texttt{id} (PK), \texttt{unlocked}, \texttt{level}, \texttt{xp} & Personajes jugables. IDs predeterminados: 1 = Ryomen Sukuna, 2 = Satoru Gojo. \\
    \midrule
    \texttt{upgrades} & \texttt{id} (TEXT PK), \texttt{level}, \texttt{purchased} & Mejoras compradas en la tienda (daño, auto-clicker, etc.). \\
    \midrule
    \texttt{skills} & \texttt{id} (TEXT PK), \texttt{character\_id}, \texttt{unlocked}, \texttt{level} & Habilidades de los personajes (Cleave, Dismantle, Fuga, Domain Expansion, etc.). \\
    \midrule
    \texttt{enemies} & \texttt{id}, \texttt{enemy\_level}, \texttt{defeated\_count} & Estado del progreso de enemigos (nivel actual y derrotados). \\
    \midrule
    \texttt{scores} & \texttt{id}, \texttt{user\_id}, \texttt{player\_name}, \texttt{game\_name}, \texttt{score\_value}, \texttt{created\_at}, \texttt{extra} & Tabla de puntuaciones compartida entre todos los juegos. \\
    \bottomrule
\end{longtable}

\subsubsection*{Aislamiento por usuario}

Los datos se aíslan por usuario mediante \textbf{bases de datos separadas}: cada usuario obtiene su propia BD con nombre \texttt{kaisen\_clicker\_<username>.db}. El constructor de \texttt{GameDataManager} recibe el username y genera el nombre de fichero correspondiente.

\begin{lstlisting}
// GameDataManager.java
String dbName = (username != null && !username.isEmpty())
        ? "kaisen_clicker_" + username + ".db"
        : AppDatabaseHelper.DATABASE_NAME;
repository = new SqlRepository(context, dbName);
\end{lstlisting}

\subsection{Base de Datos de Estadísticas: \texttt{kaisenclicker\_stats.db}}

Gestionada por \texttt{StatsDatabaseHelper} (versión 1). Base de datos independiente y específica para las estadísticas del módulo Kaisen Clicker.

\vspace{0.3cm}
\noindent\textbf{Ruta:} \texttt{kaisenclicker\_module/src/main/java/com/example/kaisenclicker/StatsDatabaseHelper.java}

\begin{longtable}{p{4cm}p{10cm}}
    \toprule
    \textbf{Columna} & \textbf{Descripción} \\
    \midrule
    \texttt{user\_id} & ID del usuario \\
    \texttt{total\_damage} & Daño total infligido \\
    \texttt{dps} & Daño por segundo (REAL) \\
    \texttt{enemies} & Enemigos normales derrotados \\
    \texttt{bosses} & Bosses derrotados \\
    \texttt{clicks} & Clicks totales \\
    \texttt{unlocked} & Personajes desbloqueados \\
    \texttt{next\_progress} & Progreso al siguiente objetivo (0--100) \\
    \bottomrule
\end{longtable}

\subsection{Capa de Acceso a Datos: \texttt{SqlRepository}}

Clase central de acceso a la BD (\textbf{811 líneas}). Provee operaciones CRUD tipadas sobre el almacén clave-valor y las tablas específicas.

\vspace{0.3cm}
\noindent\textbf{Ruta:} \texttt{kaisenclicker\_module/.../persistence/save/SqlRepository.java}

\begin{itemize}
    \item \textbf{KV Store}: \texttt{putInt/getInt}, \texttt{putLong/getLong}, \texttt{putString/getString} --- con reintentos automáticos ante fallos.
    \item \textbf{Characters}: \texttt{upsertCharacter()}, \texttt{getAllCharacters()}.
    \item \textbf{Enemies}: \texttt{setEnemyLevel()}, \texttt{getCurrentEnemyLevel()}, \texttt{setCurrentEnemyState()}.
    \item \textbf{Scores}: Lectura/escritura de puntuaciones para el leaderboard.
\end{itemize}

\subsection{Persistencia Dual: SharedPreferences + SQLite}

\texttt{GameDataManager} (\textbf{508 líneas}) implementa una estrategia de \textbf{persistencia dual}:

\begin{enumerate}
    \item Los datos se escriben \textbf{siempre} en \texttt{SharedPreferences} (escritura rápida).
    \item Simultáneamente se escriben en \texttt{SQLite} vía \texttt{SqlRepository}.
    \item Al leer, si la migración a SQL está completa, se prefiere SQLite; de lo contrario, se usa SharedPreferences como fallback.
    \item La migración automática se ejecuta en \texttt{migratePrefsToSqlIfNeeded()}.
\end{enumerate}

\subsection{Autenticación de Usuarios: \texttt{UserRepository}}

\noindent\textbf{Ruta:} \texttt{kaisenclicker\_module/.../persistence/save/UserRepository.java}

\begin{itemize}
    \item Usa la BD global (\texttt{kaisen\_clicker.db}), \textbf{no} la per-usuario.
    \item \texttt{registerUser()}: Inserta en tabla \texttt{users} con contraseña hasheada (SHA-256).
    \item \texttt{authenticateUser()}: Verifica credenciales comparando hashes.
    \item \texttt{userExists()}: Comprueba si un username ya está registrado.
\end{itemize}

\subsection{Sesión de Usuario: \texttt{SessionManager}}

\noindent\textbf{Ruta:} \texttt{app/.../auth/SessionManager.java}

Gestiona la sesión activa del usuario mediante \texttt{SharedPreferences} (\texttt{GameHubSession}):
\begin{itemize}
    \item Login/logout, username, estado (online/playing/away).
    \item Foto de perfil (URI), puntos totales, partidas jugadas.
    \item \textbf{Tracking de tiempo}: \texttt{markGameStarted()} / \texttt{markGameEnded()} para acumular segundos jugados con protección anti-corrupción (máximo 24h por sesión).
\end{itemize}

% ══════════════════════════════════════════════════════════════
\section{GIFs Animados --- Animaciones de Habilidades}

El módulo Kaisen Clicker utiliza \textbf{GIFs animados} como efectos visuales a pantalla completa cuando se activan habilidades especiales de los personajes. Los GIFs se cargan mediante la librería \textbf{Glide}.

\subsection{Archivos GIF}

\begin{longtable}{p{6.5cm}p{7.5cm}}
    \toprule
    \textbf{Fichero} & \textbf{Uso} \\
    \midrule
    \texttt{fuga\_animation.gif} & Habilidad ``Fuga'' de Ryomen Sukuna \\
    \texttt{hollow\_purple\_animation.gif} & Habilidad ``Vacío Púrpura'' de Satoru Gojo \\
    \texttt{gojo\_domain\_animation.gif} & Expansión de Dominio de Gojo (Infinite Void) \\
    \texttt{sukuna\_domain\_animation.gif} & Expansión de Dominio de Sukuna (Malevolent Shrine) \\
    \bottomrule
\end{longtable}

\noindent\textbf{Ubicación:} \texttt{kaisenclicker\_module/src/main/res/drawable/}

\subsection{Mecanismo de Reproducción}

El método \texttt{showSkillGifAnimation()} en \texttt{CampaignFragment} gestiona la reproducción:

\begin{lstlisting}
private void showSkillGifAnimation(int gifResId, int durationMs,
                                    Runnable onComplete) {
    // 1. Hacer visible el overlay (ImageView a pantalla completa)
    skillGifOverlay.setVisibility(View.VISIBLE);

    // 2. Cargar GIF con Glide
    Glide.with(this).asGif().load(gifResId).into(skillGifOverlay);

    // 3. Fade-in rapido (200ms)
    skillGifOverlay.animate().alpha(1f).setDuration(200).start();

    // 4. Tras la duracion, fade-out y ejecutar callback de dano
    mainHandler.postDelayed(() -> {
        skillGifOverlay.animate().alpha(0f).setDuration(300)
            .withEndAction(() -> {
                skillGifOverlay.setVisibility(View.GONE);
                Glide.with(this).clear(skillGifOverlay);
                if (onComplete != null) onComplete.run();
            }).start();
    }, durationMs);
}
\end{lstlisting}

\begin{itemize}
    \item \textbf{Duración típica}: 1500ms (habilidades normales), 2000ms (dominios).
    \item \textbf{Fallback}: Si Glide falla, se carga como imagen estática con \texttt{setImageResource()}.
    \item \textbf{Liberación de memoria}: Se llama a \texttt{Glide.clear()} al terminar la animación.
    \item El daño se aplica \textbf{después} de la animación (en el callback \texttt{onComplete}).
\end{itemize}

% ══════════════════════════════════════════════════════════════
\section{Arquitectura de Módulos y Navegación}

\subsection{Registro Dinámico de Juegos}

\texttt{GameRepository} (Singleton) gestiona los juegos disponibles en el hub. Cada módulo se registra en \texttt{MainActivity.registerGames()}:

\begin{lstlisting}
repo.registerGame(new Game(
    "kaisen_clicker",
    "Kaisen Clicker",
    com.example.kaisenclicker.R.drawable.kaisen_icon,
    "Haz clic para derrotar maldiciones!",
    com.example.kaisenclicker.ui.activities.MainActivity.class
));

repo.registerGame(new Game(
    "2048", "2048",
    com.example.a2048.R.drawable.icon,
    "Desliza y combina numeros!",
    com.example.a2048.MainActivity.class
));
\end{lstlisting}

El modelo \texttt{Game} encapsula: ID, nombre, icono, descripción y la \texttt{Activity.class} destino.

\subsection{Flujo de Navegación Principal}

\begin{enumerate}
    \item \textbf{LoginActivity} $\rightarrow$ Autenticación vía \texttt{UserRepository}.
    \item \textbf{MainActivity (Hub)} $\rightarrow$ Grid de juegos, perfil, menú.
    \item Al seleccionar un juego $\rightarrow$ Se lanza la \texttt{Activity} del módulo correspondiente con \texttt{EXTRA\_USERNAME}.
    \item \textbf{Al volver} $\rightarrow$ \texttt{SessionManager.markGameEnded()} acumula tiempo jugado.
\end{enumerate}

\subsection{BaseActivity --- Modo Inmersivo}

Todas las Activities del hub extienden \texttt{BaseActivity}, que oculta automáticamente la barra de navegación del sistema para ofrecer una experiencia a pantalla completa (API 30+: \texttt{WindowInsetsController}; versiones anteriores: flags \texttt{SYSTEM\_UI}).

% ══════════════════════════════════════════════════════════════
\section{Sistema de Juego: Kaisen Clicker}

\subsection{Fragmentos Principales}

\begin{longtable}{p{4.5cm}p{9.5cm}}
    \toprule
    \textbf{Fragment} & \textbf{Función} \\
    \midrule
    \texttt{CampaignFragment} & Fragmento principal del combate (\textbf{2676 líneas}). Gestiona taps, habilidades, enemigos, bosses, DPS, ultimate y efectos visuales. \\
    \texttt{ShopFragment} & Tienda de mejoras: Daño de Tap, Auto Clicker, Black Flash, Energy Boost. Costes escalados por nivel. \\
    \texttt{ChestFragment} & Sistema de cofres con probabilidad de desbloquear personajes (30\%) y recompensas de energía maldita escaladas por nivel. \\
    \texttt{CharacterInventoryFragment} & Inventario de personajes desbloqueados. \\
    \texttt{StatisticsFragment} & Pantalla de estadísticas del jugador. \\
    \bottomrule
\end{longtable}

\subsection{Sistema de Combate}

\begin{itemize}
    \item \textbf{Enemigos}: HP escalado exponencialmente: $HP = 45 \times 1.11^{(nivel-1)}$, con tope en $5 \times 10^8$.
    \item \textbf{Bosses}: Choso (dos fases), Mahito (transformación), Mahoraga (adaptación progresiva).
    \item \textbf{Auto Clicker}: Clicks automáticos cuya frecuencia depende del nivel comprado.
    \item \textbf{DPS en tiempo real}: Ventana deslizante de 3 segundos, recalculada cada 500ms.
    \item \textbf{Global ultimate}: Carga de 0 a 100\%, permite usar la Expansión de Dominio.
\end{itemize}

\subsection{Sistema de Habilidades}

Gestionado por \texttt{CharacterSkillManager}. Cada habilidad tiene tipo, cooldown, nivel máximo y parámetros de sangrado (DoT):

\begin{longtable}{p{3.5cm}p{2.5cm}p{2cm}p{5.5cm}}
    \toprule
    \textbf{Habilidad} & \textbf{Tipo} & \textbf{Cooldown} & \textbf{Efecto} \\
    \midrule
    Cleave & NORMAL\_1 & 2s & 100\% + 20\%/nivel + sangrado (5s, ticks cada 1s) \\
    Dismantle & NORMAL\_2 & 3s & \% de vida actual del enemigo (12\% base + 3\%/nivel) \\
    Fuga & NORMAL\_3 & 4s & 120\% + 35\%/nivel, golpe devastador \\
    Domain Expansion & ULTIMATE & 8s & 300\% + 60\%/nivel, ráfaga masiva \\
    \bottomrule
\end{longtable}

\noindent Los cooldowns se reducen un 10\% por cada nivel adicional de la habilidad.

\subsection{Personajes Jugables}

\begin{longtable}{p{3.5cm}p{1cm}p{9.5cm}}
    \toprule
    \textbf{Personaje} & \textbf{ID} & \textbf{Habilidades específicas} \\
    \midrule
    Ryomen Sukuna & 1 & Cleave, Dismantle, Fuga, Domain Expansion (Malevolent Shrine) \\
    Satoru Gojo & 2 & Amplificación Azul, Ritual Inverso Rojo, Vacío Púrpura, Domain (Infinite Void) \\
    \bottomrule
\end{longtable}

\noindent Gojo: su dominio activa un estado donde las habilidades no tienen cooldown y Vacío Púrpura es gratuito.

% ══════════════════════════════════════════════════════════════
\section{Sistema de Juego: 2048}

\subsection{Motor del Juego: \texttt{GameEngine}}

Implementación clásica del juego 2048 con una matriz \texttt{int[4][4]}:

\begin{itemize}
    \item \textbf{Spawn}: Al iniciar, dos casillas aleatorias. Tras cada movimiento válido, aparece un 2 (90\%) o un 4 (10\%).
    \item \textbf{Movimientos}: \texttt{moveLeft()}, \texttt{moveRight()}, \texttt{moveUp()}, \texttt{moveDown()}.
    \item \textbf{Lógica}: \texttt{compressAndMerge()} comprime y fusiona una línea, sumando al score al fusionar.
    \item \textbf{Persistencia}: Score y best score se guardan en \texttt{kv\_store} de la BD global con claves \texttt{2048\_score}, \texttt{2048\_best\_score}, \texttt{2048\_moves}, \texttt{2048\_seconds}.
    \item \textbf{Detección táctil}: \texttt{OnSwipeTouchListener} detecta gestos de deslizamiento.
\end{itemize}

% ══════════════════════════════════════════════════════════════
\section{Sistema de Desafíos}

La clase \texttt{ChallengesActivity} define logros que el usuario puede completar para ganar puntos:

\begin{longtable}{p{3.5cm}p{4cm}p{2cm}p{3cm}}
    \toprule
    \textbf{ID} & \textbf{Nombre} & \textbf{Objetivo} & \textbf{Recompensa} \\
    \midrule
    \texttt{first\_game} & Primer Paso & 1 partida & 50 puntos \\
    \texttt{five\_games} & Jugador Habitual & 5 partidas & 100 puntos \\
    \texttt{ten\_games} & Veterano & 10 partidas & 200 puntos \\
    \texttt{hundred\_points} & Primeros Puntos & 100 pts & 50 puntos \\
    \texttt{thousand\_points} & Mil Puntos & 1.000 pts & 150 puntos \\
    \texttt{five\_thousand} & Maestro del GameHub & 5.000 pts & 500 puntos \\
    \texttt{twentyfive\_games} & Adicto al Juego & 25 partidas & 300 puntos \\
    \texttt{try\_all\_games} & Explorador & 2 juegos & 100 puntos \\
    \bottomrule
\end{longtable}

Cada \texttt{Challenge} calcula su progreso como porcentaje: $\text{progreso} = \min\left(100,\, \frac{\text{actual} \times 100}{\text{objetivo}}\right)$.

% ══════════════════════════════════════════════════════════════
\section{Leaderboard y Puntuaciones}

\begin{itemize}
    \item \textbf{LeaderboardActivity}: Pantalla con tabs para alternar entre las stats de Kaisen Clicker y 2048.
    \item \textbf{Kaisen Clicker}: Muestra nivel de enemigo, clicks totales, daño total, enemigos/bosses derrotados, energía maldita, nivel de personaje y tiempo jugado.
    \item \textbf{2048}: Muestra score actual, mejor score, movimientos y tiempo.
    \item \textbf{ScoresListActivity}: Historial de puntuaciones con búsqueda, ordenación y swipe-to-delete. Usa \texttt{Cursor} + \texttt{RecyclerView}.
    \item Los datos de 2048 se leen del \texttt{SqlRepository} con claves \texttt{2048\_*} en la BD per-usuario.
\end{itemize}

% ══════════════════════════════════════════════════════════════
\section{Dependencias Principales}

\begin{longtable}{p{5cm}p{9cm}}
    \toprule
    \textbf{Dependencia} & \textbf{Uso} \\
    \midrule
    \texttt{appcompat} & Compatibilidad hacia atrás de Activities/Fragments \\
    \texttt{material} & Componentes Material Design (CardView, Buttons, etc.) \\
    \texttt{constraintlayout} & Layouts responsivos \\
    \texttt{recyclerview} & Listas eficientes (juegos, puntuaciones, desafíos) \\
    \texttt{core-splashscreen} & Pantalla de splash nativa (API 31+) \\
    \texttt{gridlayout} & Grid para la pantalla principal \\
    \texttt{Glide} & Carga y renderizado de GIFs animados \\
    \bottomrule
\end{longtable}

% ══════════════════════════════════════════════════════════════
\section{Resumen de Rutas Clave}

\begin{longtable}{p{4cm}p{10cm}}
    \toprule
    \textbf{Componente} & \textbf{Ruta relativa} \\
    \midrule
    BD principal & \texttt{kaisenclicker\_module/.../persistence/save/AppDatabaseHelper.java} \\
    Repositorio SQL & \texttt{kaisenclicker\_module/.../persistence/save/SqlRepository.java} \\
    Gestor de datos & \texttt{kaisenclicker\_module/.../persistence/save/GameDataManager.java} \\
    Repositorio usuarios & \texttt{kaisenclicker\_module/.../persistence/save/UserRepository.java} \\
    BD estadísticas & \texttt{kaisenclicker\_module/.../StatsDatabaseHelper.java} \\
    Sesión & \texttt{app/.../auth/SessionManager.java} \\
    GIFs animados & \texttt{kaisenclicker\_module/src/main/res/drawable/*.gif} \\
    Motor 2048 & \texttt{2048\_module/.../GameEngine.java} \\
    Hub principal & \texttt{app/.../MainActivity.java} \\
    Desafíos & \texttt{app/.../ChallengesActivity.java} \\
    Leaderboard & \texttt{app/.../leaderboard/LeaderboardActivity.java} \\
    \bottomrule
\end{longtable}

\end{document}
